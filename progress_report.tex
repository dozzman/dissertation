% preamble
\documentclass[a4paper]{article}
\usepackage{nameref}

% body
\begin{document}

\input{./progress_report_title.tex}

\section{Introduction}
\label{Introduction}
My project, as a quick overview, is to get parallel OCaml code running on the JVM with the use of OCaml-Java and LWT. As a summary of the below, I feel as though the project is moving along well although very slightly behind schedule, with the greatest difficulty being getting to a point where OCaml-Java and LWT are managing to communicate and compile programs, which has been achieved.

\section{Current State of the Project}
\label{Current State of the Project}
The project is at a point where LWT and OCaml-Java are interfacing and as such I am able to compile and run programs written in LWT with OCaml-Java. This corresponds to one milestone behind my intended progress, however the next milestone is fairly close to completion. Getting OCaml-Java and LWT to compile and interface correctly was a much harder task than I had initially expected since there are lots of intermediate `helper' programs which LWT relies on -- such as Findlib, Camlp4 and OCamlbuild -- which are mainly used for building. Attempting to make these programs function correctly under OCaml-Java would have been an increasingly difficult task, resulting in a lot more code manipulation in various code bases. In order to remove remove the use of Findlib (a library which provides easier compilation of OCaml software by automatically pulling in any necessary dependencies) I manually converted the results of the `ocamlfind' commands to OCaml-Java compatible versions and dumped the results into a new build script. As a result of these unforeseen complications, some small changes must be made to my schedule to correctly adjust my time allocation. The new schedule follows in section `\nameref{Ongoing Plan}'.

\section{What's Left}
\label{What's Left}
The scheduler still needs to be written which will -- at the early stages -- provide an disproportionate boost to functionality and should be relatively straight forward to extend to complete the parallel aspect of the project. Benchmarking and testing still also needs to be done.

\section{Ongoing Plan}
\label{Ongoing Plan}
Currently my focus is with Java, writing the scheduler and attempting to link it to LWT's code. LWT is written in such a way as to make this a more straightforward task than I had expected. The multiplexer used for asynchronous I/O is actually a pluggable component called LWT Engine, complete with abstract class and virtual methods so that the external select command is all that needs to be linked to the OCaml code. This means implementing the OCaml section of the multiplexer will be a little simpler. I'm also currently looking into asynchronous programming techniques as this is not something I have done much of -- especially in the low level.

\subsection{Ammended Timetable}
\begin{description}
\item[30$^{st}$ Jan -- 2$^{nd}$ Mar] Continued scheduler development.\\
Milestone: Number of Java threads scales with the number of cores present on the system.

\item[3$^{rd}$ Mar -- 12$^{th}$ Mar] begin conversion of paralel processing benchmarks.\\
Milestone: Most of benchmarking software conversion complete, fixed relevant bugs which may arise from testing under OCaml-Java.

\item[13$^{th}$ Mar -- 2$^{nd}$ April] Testing with benchmarks, finishing dissertation evaluation and write-up.\\
Milestone: Relevant scalability graph data accumulated.

\item[3$^{rd}$ April -- 23$^{rd}$ April] Finalising dissertation write-up.\\
Milestone: Dissertation complete!
\end{description}

\end{document}

